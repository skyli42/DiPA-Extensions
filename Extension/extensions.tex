\documentclass[12pt]{article}

\usepackage[shortlabels]{enumitem} 
\usepackage{amsmath,amsfonts,amssymb,amsthm,bm,mathrsfs}
\usepackage{fancyhdr}
\usepackage[margin=1in]{geometry}
\usepackage{parskip}
\usepackage{tikz}
\usepackage{xcolor}

\newcommand{\NN}{\mathbb{N}}
\newcommand{\ZZ}{\mathbb{Z}}
\newcommand{\QQ}{\mathbb{Q}}
\newcommand{\RR}{\mathbb{R}}
\newcommand{\CC}{\mathbb{C}}
\newcommand{\PP}{\mathbb{P}}
\newcommand{\EE}{\mathbb{E}}
\newcommand{\notimplies}{\;\not\!\!\!\implies}
\newcommand{\gguard}{\texttt{insample} \geq \texttt{x}}
\newcommand{\lguard}{\texttt{insample} < \texttt{x}}
\providecommand{\floor}[1]{ \lfloor #1 \rfloor }
\newcommand{\dist}{\text{dist}}
\newtheorem{thm}{Theorem}[section]
\newtheorem{lemma}[thm]{Lemma}
\newtheorem{prop}[thm]{Proposition}
\newtheorem{cor}[thm]{Corollary}
\theoremstyle{definition}
\newtheorem{defn}[thm]{Definition}
\newtheorem{examp}[thm]{Example}
\newtheorem{conj}[thm]{Conjecture}
\newtheorem{rmk}[thm]{Remark}
\let\bar\overline
\let\tilde\widetilde

% morphic arrows
\let\into\hookrightarrow
\let\onto\twoheadrightarrow
\newcommand{\isto}{\stackrel{\sim}{\smash{\longrightarrow}\rule{0pt}{0.4ex}}} 
\graphicspath{ {./} }
\bibliographystyle{plain} 


\newcommand{\bigo}{\mathcal{O}}

\begin{document}

\section{DiPA*}

\begin{defn} Fix parameters $\epsilon, N$. Let $C$ be the guard conditions $\{\texttt{true}, \texttt{insample}\geq \texttt{x}, \texttt{insample} < \texttt{x}, n \geq N\}$ A \textbf{DiP* automaton} (DiPA*) $\mathcal{A}$ is defined as the tuple $\mathcal{A} = (Q, \Sigma, \Gamma, q_0, X, P, \delta)$, where:
\begin{itemize}
	\item $Q = $ finite set of states; partitioned into input states $Q_{in}$ and non-input states $Q_{non}$
	\item $\Sigma$ is the input alphabet (taken to be $\RR$)
	\item $\Gamma$ is a finite output alphabet
	\item $q_0\in Q$ is the starting state
	\item $X$= \{\texttt{x}, \texttt{insample}, \texttt{insample'}, \texttt{n}\} is a set of variables. $\texttt{x}, \texttt{insample}, \texttt{insample'} \in \RR$; $\texttt{n} \in \NN$ and is initialized to 0.
	\item $P: Q \to \QQ^{\geq 0} \times \QQ \times \QQ^{\geq 0} \times \QQ$ describing the parameters for sampling from Laplace distributions at each state.
	\item $\delta: (Q \times C) \to Q\times (\Gamma \cup \{\texttt{insample}, \texttt{insample'}\}\cup \{\phi\}) \times \{\texttt{true}, \texttt{false}\} \times \{0, 1\}$ is the transition function (technically a relation) that defines what state to transition to, what symbol or real value to output, whether or not $x$ is assigned to, and whether or not $n$ is incremented based on the current state and transition guard.
\end{itemize}
\end{defn}

There are certain conditions that $\delta$ must satisfy; these are almost all the same as the restrictions on transition functions of DiPA, but with some slight modifications and one major addition (marked in \textcolor{blue}{blue}):
\begin{itemize}
	\item \textbf{Determinism:} For any state $q\in Q$, if $\delta(q, \texttt{true})$ is defined, then $\delta(q, \texttt{insample} \geq \texttt{x})$, $\delta(q, \texttt{insample} < \texttt{x})$, \textcolor{blue}{and $\delta(q, n\geq N)$} are undefined. \textcolor{blue}{In addition, if two guard conditions are simultaneously true, then the automaton will follow the transition with guard $n \geq N$}.

	\item \textbf{Output Distinction:} For any state $q\in Q$, if $\delta(q, \gguard) = (q_1, o_1, b_1, i_1)$ and $\delta(q, \lguard) = (q_2, o_2, b_2, i_2)$, then $o_1\neq o_2$ and at least one of $o_1\in \Gamma$ and $o_2\in \Gamma$ is true. \textcolor{blue}{In addition, $o_1\neq \phi$ and $o_2\neq \phi$ and if $\delta(q, n\geq N) = (q', o', b', i')$, then $o' = \phi$, i.e., the $\phi$ output symbol is reserved for transitions with guard $n\geq N$, which must output $\phi$}.

	\item \textbf{Initialization:} The initial state $q_0$ has only one outgoing transition of the form $\delta(q_0, \texttt{true}) = (q, o, \texttt{true}, i)$.

	\item \textbf{Non-input transition:} From any $q\in Q_{non}$, if $\delta(q, c)$ is defined, then $c=\texttt{true}$.

	\item \textcolor{blue}{\textbf{Control Flow Separation:} Consider the underlying graph $G$ of $\mathcal{A}$. For all states $q\in Q$, if $\delta(q, n\geq N) = (q', o, b, i)$, let $G'$ be the graph of $G$ with the edge corresponding to the transition with guard $n\geq N$ from $q$ removed. Then $q$ and $q'$ must be in different connected components of $G'$}.

\end{itemize}

Note that the \textbf{control flow separation} condition implies that no cycle in $G$ can contain an edge that corresponds to a transition with guard $n\geq N$. In addition, determinism combined with control flow separation imply that no two transitions (i.e. transitions with different guards) can be from some state $q$ to the same state $q'$. 

\subsection{TODO: path probabilities}

tldr $n\geq N$ transitions $\to$ same probability as $\texttt{true}$ if guard condition is met, and $\gguard, \lguard$ transitions also need to take this into account.

\section{Violations of Differential Privacy}

\begin{defn}
A \textbf{bounded} cycle $C$ in a DiPA* $\mathcal{A}$ is a cycle in $\mathcal{A}$ where there exists at least one transition $(q', \sigma, t, 1)$ (i.e. $\texttt{n}$ gets incremented) and there exists some $q\in Q$ (``exit state'') in the cycle such that $f(q, n \geq N) = (q', \sigma, t, i)$ where $q'$ is not in the cycle. Otherwise, the cycle is \textbf{unbounded}. 
\end{defn}

\begin{defn}
A cycle $C$ with an exit state with transition $n\geq N$ is a \textbf{trivially exiting} cycle if, for \textit{all} paths $\rho = q_0q_1...q_m)$ from the start state to a state $q_m \in C$, at least $N$ transitions $q_i \to q_{i+1}$ are increment transitions or some transition $q_i \to q_{i+1}$ has guard $n\geq N$.
\end{defn}


\begin{prop} If a DiPA* $\mathcal{A}$ has a reachable unbounded leaking non-trivially exiting cycle, then it is not differentially private.
\end{prop}

\begin{proof} 

Let $C = a_1a_2...a_{n-1}a_n; a_1=a_n$ be such a cycle in $\mathcal{A}$. We will reduce the analysis to an analogous DiPA.

\textbf{Case 1:} $C$ does not have an exit state.

Consider an abstract path $\eta = q_0\sigma_0q_1...q_{m+n-1}\sigma_{m+n-1}q_m$ such that $a_1...a_n=q_m...q_{m+n}$ (i.e. the last $n$ states of the path are the cycle $C$). 

For $\ell > 0$, let $\eta_\ell$ be the path $\eta_\ell = q_0\sigma_0q_1\sigma_1...q_{m+\ell n-1}\sigma_{m+\ell n-1}q_{m+\ell n}$ such that $q_k = q_{k-n}$ and $\sigma_k = \sigma_{k-n}$ for all $m+n\leq k\leq m + \ell n$. This is the path $\eta$ with the cycle $C$ repeated $\ell$ times. 
Note that because $C$ has no exit state, for all states $a_i\in C$, all transitions from $a_i$ have a guard that is \textit{not} $n\geq N$. This means that the path $\eta_\ell$ in $\mathcal{A}$ exists for all $\ell>0$. 
Thus, the same input sequences $\alpha_\ell$ and $\beta_\ell$ as described in Lemma 6 of~\cite{chadhaLinearTimeDecidability2021} are witnesses to a violation of differential privacy. In particular, the same analysis holds because there is some fixed number $f$ such that $\eta_\ell$ has at most $f$ transitions with guard $n\geq N$, even as $\ell$ can vary arbitrarily.

\textbf{Case 2:} Suppose that $C$ has no increment transition.

Because $C$ is non-trivially exiting, there exists some path $\eta = q_0\sigma_0q_1\cdots q_{m+n-1}\sigma_{m+n-1}q_m$ such that $a_1\cdots a_m = q_m\cdots q_{m+n}$ and at $q_m = a_1$, $n < N$.

As in Case 1, for $\ell > 0$, consider $\eta_\ell = q_0\sigma_0q_1\sigma_1\cdots q_{m+\ell n-1}\sigma_{m+\ell n-1}q_{m+\ell n}$ such that $q_k = q_{k-n}$ and $\sigma_k = \sigma_{k-n}$ for all $m+n\leq k\leq m + \ell n$. 
Because there are no increment transitions in $C$, $\forall 0\leq i\leq \ell n$, $n<N$ at state $q_i$. So for all states $a_i\in C$, a transition from $a_i$ with guard $n\geq N$ will never be taken by $\mathcal{A}$. As before, then, the path $\eta_\ell$ in $\mathcal{A}$ exists for all $\ell > 0$, so $\alpha_\ell$ and $\beta_\ell$ from Lemma 6 of~\cite{chadhaLinearTimeDecidability2021} are witnesses to a violation of differential privacy.

\end{proof}

\begin{defn}
	A bounded pair is a pair of cycles $(C, C')$ such that at least one of $C$ and $C'$ is a bounded cycle. Similarly, a pair of cycles $(C, C')$ is trivially exiting if at least one of $C$ and $C'$ are trivially exiting.
\end{defn}

\begin{prop}
	If a DiPA* $\mathcal{A}$ has an unbounded non-trivially exiting leaking pair of cycles $(C, C')$ where $C$ is reachable, then it is not differentially private.
\end{prop}

\begin{prop}
	If a DiPA* $\mathcal{A}$ has a reachable unbounded non-trivially exiting disclosing cycle, then it is not differentially private.
\end{prop}

\begin{defn}
	A unbounded and non-trivially exiting privacy violating path is a path $\rho$ of length $n$ in a DiPA* $\mathcal{A}$ such that one of the following hold:

\begin{itemize}
	\item $\texttt{tail}(\rho)$ is an $\texttt{AG}$-path such that $\texttt{last}(\rho)$ is in a unbounded non-triviallt exiting $\texttt{G}$-cycle and the 0th transition is an assignment transition that outputs $\texttt{insample}$ (or $\texttt{AL}, \texttt{L}$-cycle, respectively)
	\item $\rho$ is an $\texttt{AG}$-path such that $\texttt{first}(\rho)$ is in a unbounded non-trivially exiting $\texttt{G}$-cycle and the 0th transition has guard $\texttt{insample < x}$ and outputs $\texttt{insample}$ (or similar with $\texttt{L}$-cycles)
	\item $\rho$ is an $\texttt{AG}$-path such that $\texttt{first}(\rho)$ is in a unbounded non-trivially exiting $\texttt{L}$-cycle and the last transition has guard $\texttt{insample} \geq \texttt{x}$ and outputs $\texttt{insample}$ (or similar)

\end{itemize}
\end{defn}

\begin{prop}
	If a DiPA* $\mathcal{A}$ has an unbounded non-trivially exiting privacy violating path, then it is not differentially private.
\end{prop}

If this text is still here, I didn't have time to write out the details, but propositions 2.5, 2.6, and 2.8 are all proved extremely simlarly to proposition 2.3.

\begin{defn} 
	A DiPA* $\mathcal{A}$ is well-formed if $\mathcal{A}$ has no unbounded non-trivially exiting leaking cycles, unbounded non-trivially exiting leaking pairs, unbounded non-trivially exiting disclosing cycles, or unbounded non-trivially exiting privacy violating paths.
\end{defn}

\begin{thm} 
	If a DiPA* is not well-formed, then it is not differentially private.
\end{thm}
\begin{proof}
	Follows from propositions 2.3, 2.5, 2.6, and 2.8.
\end{proof}

\section{Proving Differential Privacy}

Let $t = (p, c, q, o, b)$ be a transition in a DiPA* $\mathcal{A}$ with parameters $\epsilon, N$, where the transition is from $p \to q$, $c$ is the guard of the transition, $o$ is the output of the transition, and $b$ is whether or not it assigns to $x$.
Let $d, \mu$ be the parameters for sampling $\texttt{insample}$ and $d', \mu'$ be the parameters for sampling $\texttt{insample'}$ at state $p$. $t$ is a critical transition if it is not in a cycle in $\mathcal{A}$ or, if it is in a cycle $C$, that $C$ trivially exits. $t$ is an input transition if $p$ is an input state.

We will assign a cost to each transition as follows:

\[
	cost(t) = \begin{cases}
		d & t \text{ is a critical non-input transition}\\
		2d & t \text{ is a critical input transition and } o \neq \texttt{insample'}\\
		2d+d' & t \text{ is a critical input transition and }o = \texttt{insample'}\\
		dN & t \text{ is a non-input transition on a bounded cycle}\\
		2dN & t \text{ is a critical input transition on a bounded cycle and } o \neq \texttt{insample'}\\
		(2d+d')N & t \text{ is a critical input transition on a bounded cycle and } o = \texttt{insample'}\\
		0 & \text{otherwise}
	\end{cases}
\]

Note to self: this may not be necessary? if each path already accounts for this


\begin{thm} 
	If a DiPA* is well-formed, then it is differentially private.
\end{thm}

\begin{proof}



Let $\mathcal{A}^* = (Q, \Sigma, \Gamma, q_0, X^*, P^*, \delta^*)$ be a well-formed DiPA* with parameters $\epsilon$ and $N$. 

Let $G = \{\texttt{true}, \gguard, \lguard \}$ be the set of guard conditions for DiPAs.

For $n\in \NN$, let $[N] = \{0, 1, \cdots , N\}$.
Construct the DiPA $\mathcal{A} = (Q \times [N], \Sigma, \Gamma\cup \{\phi\}, (q_0, 0), X, P, \delta)$ as follows:



For each state $q \in Q^*$:

For $g\in G$, if $\delta^*(q, g) = (q', \sigma, \mathbf{b}, x)$ is defined, then for all $k \in [N-1]$, define the transition
\[
	\delta((q, k), g) = ((q', k+x), \sigma, \mathbf{b})	
\]

If $\delta^*(q, n\geq N) = (q', \sigma, \mathbf{b}, x)$ is defined, then define the transition \[
	\delta((q, N), \texttt{true}) = ((q', N), \sigma, \mathbf{b})
\]

Otherwise if $\delta^*(q, g) = (q', \sigma, \mathbf{b}, x)$ is defined, then define the transition
\[
	\delta((q, N), g) = ((q', N), \sigma, \mathbf{b})
\]

Note that we need to separate out the $k = N$ case to satisfy the condition of determinism.

Intuitively, at state $(q, k)$ in $\mathcal{A}$, $k$ will track the value of $n$ in $\mathcal{A}^*$ (since everything above $N$ is treated the same, we compress all of those values together).



First, \textbf{for every bounded cycle $C^*$ in $\mathcal{A^*}$:}
We will ``unroll'' $C^*$ to loop for $N$ iterations.

For each state $q^*\in C^*$:

For $0\leq k \leq N$, $q^{*(k)} \in Q$; i.e. we create $N+1$ copies of $C^*$. 


For guard conditions $g\in G$ and if $0\leq k < N$, we define new transitions as follows:

If $\delta^*(q^*, g) = (q', \sigma, \mathbf{b}, x)$ is defined, then
if $x = 0$,  
\[\delta(q^{*(k)}, g) = \begin{cases}
	(q'^{(k)}, \sigma, \mathbf{b}) &\text{ if } q' \text{ is in \textit{any} bounded cycle}\\
	(q', \sigma, \mathbf{b}) & \text{ otherwise}
\end{cases}\]


Otherwise if $x=1$, then 
\[\delta(q^{*(k)}, g) = \begin{cases}
	(q'^{(k+1)}, \sigma, \mathbf{b}) &\text{ if } q' \text{ is in \textit{any} bounded cycle}\\
	(q', \sigma, \mathbf{b}) &  \text{ otherwise}
\end{cases}\]

Now consider when $k=N$. If $\delta^*(q^*, n\geq N) = (q', \sigma, \mathbf{b}, x)$ is defined, 
\[\delta(q^{*(N)}, \texttt{true}) = \begin{cases}
	(q'^{(N)}, \sigma, \mathbf{b}) &\text{ if } q' \text{ is in \textit{any} bounded cycle}\\
	(q', \sigma, \mathbf{b}) & \text{ otherwise}
\end{cases}\]

Otherwise for $g\in G$, if $\delta^*(q^*, g) = (q', \sigma, \mathbf{b}, x)$ is defined,
\[\delta(q^{*(N)}, g) = \begin{cases}
	(q'^{(N)}, \sigma, \mathbf{b}) &\text{ if } q' \text{ is in \textit{any} bounded cycle}\\
	(q', \sigma, \mathbf{b}) & \text{ otherwise}
\end{cases}\]

Additionally, \textbf{for every unbounded trivially exiting cycle $C^*$ in $\mathcal{A}$:}

For each $q^* \in C^*$, $q^*\in Q$. 
For guard conditions $g\in G$, if $\delta^*(q^*, g) = (q', \sigma, \mathbf{b}, x)$, then $\delta(q^*, g) = (q', \sigma, \mathbf{b})$. Otherwise, if $\delta^*(q^*, n\geq N) = (q', \sigma, \mathbf{b}, x),$ then $\delta(q^*, \texttt{true}) = (q', \sigma, \mathbf{b})$. As before, also remove all non-true transitions from $q^*$.

\textbf{If $q^*$ is not in a bounded or trivially exiting cycle}, then $q^* \in Q$. 

Let $a$ be the minimum number of increment transitions over \textit{all} paths from the start state $q^*_0$ to $q^*$ in $\mathcal{A}^*$.

Then for each guard $g\in G$, if $\delta^*(q^*, g) = (q', \sigma, \mathbf{b}, 0)$, then 
\[\delta(q^*, g) = \begin{cases}
(q'^{(\min\{N, a\})}, \sigma, \mathbf{b}) & \text{ if }q' \text{ is in a  bounded cycle } C \\
(q', \sigma, \mathbf{b}) & \text{ if }q' \text{ is not in a bounded cycle}
\end{cases}\]

Otherwise if $\delta^*(q^*, g) = (q', \sigma, \mathbf{b}, 1)$, then 
\[\delta(q^*, g) = \begin{cases}
(q'^{((\min\{N, a+1\})}, \sigma, \mathbf{b}) & \text{ if }q' \text{ is in a  bounded cycle } C \\
(q', \sigma, \mathbf{b}) & \text{ if }q' \text{ is not in a bounded cycle}
\end{cases}\]

For each state $q\in Q$, let $P(q) = P(q^*)$ for the corresponding state $q^*\in Q^*$.


\begin{lemma}
	If $\mathcal{A}^*$ is well-formed, then $\mathcal{A}$ is well-formed.	
\end{lemma}

This follows from the following lemma\footnote{TODO: check if this connection needs to be elaborated on}: 

\begin{lemma}
	If there exists a cycle $C$ in $\mathcal{A}$, then there exists an unbounded non-trivially exiting cycle $C^*$ in $\mathcal{A}^*$.
\end{lemma}

\begin{proof}

	Let $C = a_0a_1\cdots a_{m-1}a_0$ be a cycle in $\mathcal{A}$. Let $T$ be the set of transitions in $\mathcal{A}$ and let $T^*$ be the set of transitions in $\mathcal{A}^*$ where each transition is described by a 5-tuple $(p, c, q, o, b)$ as before.
	
	Note that by the construction of $\mathcal{A}$, for each state $q\in Q$, there exists some corresponding state $q^*\in Q^*$. Let $f_q: Q\to Q^*$ be a function describing this correspondence. Similarly, note that each transition $t = (p, c, q, o, b)$ corresponds to a transition $t^*$ in $\mathcal{A}^*$ and let $f_t:T\to T^*$ describe this correspondence ($f_q$ and $f_t$ are not necessarily injective nor surjective). 
	
	Let $a_i, a_{i+1} \in C$ (for convenience, assume that $a_{i+1}$ is actually $a_{i+1 \mod m}$). Let $t_i$ be the transition $(a_i, c, a_{i+1}, o, b)$ (recall that $t_i$ is unique because of determinism and control flow separation). Note that $f_t(t_i) = (f_q(a_i), c', f_q(a_{i+1}), o', b')$ from the construction of $\mathcal{A}$.

	Then the path $C^* = f_q(a_0)f_q(a_1)\cdots f_q(a_{m-2})f_q(a_{m-1})f_q(a_0)$ exists in $\mathcal{A}^*$ and is a cycle.

	Suppose that $C^*$ is a bounded cycle for the sake of contradiction. Then by construction of $\mathcal{A}$, all $a_i$ must be of the form $a_i = f_q{(a_i)}^{(x)}$ for some $x$. In addition, note that there cannot\footnote{Unsure if this is trivial or needs to be elaborated on} be any transitions $\mathcal{A}$ from a state $q_1^{(x)}$ to a state $q_2^{(y)}$ for $y < x$. Let $a_0 = f_q(a_0)^{(k)}$ for some fixed $k$. 
	Then, $a_{m-1}=f_q{(a_{m-1})}^{(k)}$. However, this is impossible, since there must be some increment transition in $C^*$ such that for some $a_i$, $a_i = f_q(a_i)^{(k+1)}$ because $C^*$ is a bounded cycle. Thus, $C^*$ cannot be a bounded cycle.

	Now suppose that $C^*$ is a trivially exiting cycle for the sake of contradiction. Let $f_q(a_i)$ be an exit state of $C^*$. Then $\delta(a_i, \texttt{true}) = (a_{i+1} = q', \sigma, \mathbf{b})$ for some $q'$ outside of $C^*$ by definition of an exit state and by construction of $\mathcal{A}$. 
	However, it is then impossible for $C$ to be a cycle because of control flow separation, since it would imply that, in the graph of $\mathcal{A}^*$ with the $f_q(a_i)\to f_q(q')$ transition removed, the component that contains $f_q(a_i)$ and the component that contains $f_q(q')$ are the same. Thus, $C^*$ is non-trivially exiting.
\end{proof}

\begin{lemma}
	If $\mathcal{A}$ is $wt(\mathcal{A})$-DP, then $\mathcal{A}^*$ is $wt(\mathcal{A})$-DP.
\end{lemma}


\begin{proof}
	Let $\rho^*, \rho'^*$ be two equivalent paths on adjacent inputs in $\mathcal{A}^*$. We will show that there exist paths $\rho, \rho'$ in $\mathcal{A}$ such that $\frac{\PP[\epsilon, \rho^*]}{\PP[\epsilon, \rho'^*]}\leq \frac{\PP[\epsilon, \rho]}{\PP[\epsilon, \rho']}$.


	sketch: for each state traversed in $\rho^*$, there is an equivalent state traversed in $\rho$, except for the fact that $\mathcal{A}$ may possibly enter bounded cycles ``earlier' (i.e. with smaller $n$) than $\mathcal{A}^*$ does. 

	



\end{proof}

Lemmas 3.2 and 3.4 together prove the theorem. 
\end{proof}

\bibliography{./dipalibrary}

\end{document} 