\documentclass[12pt]{article}

\usepackage[shortlabels]{enumitem} 
\usepackage{amsmath,amsfonts,amssymb,amsthm,bm,mathrsfs}
\usepackage{fancyhdr}
\usepackage[margin=1in]{geometry}
\usepackage{parskip}
\usepackage{tikz}
\usepackage{xcolor}

\newcommand{\NN}{\mathbb{N}}
\newcommand{\ZZ}{\mathbb{Z}}
\newcommand{\QQ}{\mathbb{Q}}
\newcommand{\RR}{\mathbb{R}}
\newcommand{\CC}{\mathbb{C}}
\newcommand{\PP}{\mathbb{P}}
\newcommand{\EE}{\mathbb{E}}
\newcommand{\notimplies}{\;\not\!\!\!\implies}
\newcommand{\gguard}{\texttt{insample}\geq \texttt{x}}
\newcommand{\lguard}{\texttt{insample} < \texttt{x}}
\newcommand{\gaguard}{n<N \text{ AND } \texttt{insample} \geq \texttt{x}}
\newcommand{\laguard}{n<N\text{ AND }\texttt{insample} < \texttt{x}}
\newcommand{\itgguard}{\texttt{input}\neq\tau \text{ AND } \texttt{insample} \geq \texttt{x}}
\newcommand{\itlguard}{\texttt{input}\neq\tau \text{ AND }\texttt{insample} < \texttt{x}}

\providecommand{\floor}[1]{ \lfloor #1 \rfloor }
\newtheorem{thm}{Theorem}[section]
\newtheorem{lemma}[thm]{Lemma}
\newtheorem{prop}[thm]{Proposition}
\newtheorem{cor}[thm]{Corollary}
\theoremstyle{definition}
\newtheorem{defn}[thm]{Definition}
\newtheorem{examp}[thm]{Example}
\newtheorem{conj}[thm]{Conjecture}
\newtheorem{rmk}[thm]{Remark}
\newtheorem{clm}[thm]{Claim}

\newcommand{\isto}{\stackrel{\sim}{\smash{\longrightarrow}\rule{0pt}{0.4ex}}} 
\graphicspath{ {./} }
\bibliographystyle{plain} 


\newcommand{\bigo}{\mathcal{O}}

\begin{document}

\section{DiPA with a Counter}
\subsection{DiPA* and Definitions}
\begin{defn} Fix parameters $\epsilon, N$. Let $C$ be the guard conditions $\{n<N,\texttt{true}$, $\gguard$,$\lguard$, $\gaguard, \laguard, n\geq N\}$ A \textbf{DiP* automaton} (DiPA*) $\mathcal{A}$ is defined as the tuple $\mathcal{A} = (Q, \Sigma, \Gamma, q_0, X, P, \delta)$, where:
\begin{itemize}
	\item $Q = $ finite set of states; partitioned into input states $Q_{in}$ and non-input states $Q_{non}$
	\item $\Sigma$ is the input alphabet (taken to be $\RR$)
	\item $\Gamma$ is a finite output alphabet
	\item $q_0\in Q$ is the starting state
	\item $X$= \{\texttt{x}, \texttt{insample}, \texttt{insample'}, \texttt{n}\} is a set of variables. $\texttt{x}, \texttt{insample}, \texttt{insample'} \in \RR$; $\texttt{n} \in \NN$ and is initialized to 0.
	\item $P: Q \to \QQ^{\geq 0} \times \QQ \times \QQ^{\geq 0} \times \QQ$ describing the parameters for sampling from Laplace distributions at each state.
	\item $\delta: (Q \times C) \to Q\times (\Gamma \cup \{\texttt{insample}, \texttt{insample'}\}\cup \{\phi\}) \times \{\texttt{true}, \texttt{false}\} \times \{0, 1\}$ is the transition function (technically a relation) that defines what state to transition to, what symbol or real value to output, whether or not $x$ is assigned to, and whether or not $n$ is incremented based on the current state and transition guard.
\end{itemize}
\end{defn}

There are certain conditions that $\delta$ must satisfy; these are almost all the same as the restrictions on transition functions of DiPA, but with some slight modifications and one major addition (marked in \textcolor{blue}{blue}):
\begin{itemize}
	\item \textbf{Determinism:} 
	\textcolor{blue}{If $\delta(q, \texttt{true})$ is defined, then no other transitions out of $q$ can be defined. Additionally, at most one of $\delta(q,$ $\gguard)$ and }\\
	\textcolor{blue}{$\delta(q,$$\gaguard)$ can be defined and at most one of $\delta(q,$ $\lguard)$ and $\delta(q,$ $\laguard)$ can be defined.} 
	
	\textcolor{blue}{If $\delta(q, n<N)$ is defined, then $\delta(q, \gaguard)$ and\\$\delta(q, \laguard)$ are not defined. Additionally, if any of $\delta(q, n<N)$, $\delta(q, \laguard)$, or $\delta(q, \gaguard)$ are defined, then $\delta(q, n \geq N)$ must be defined as well. 
	Finally, if $\delta(q, n\geq N)$ is defined, then $\delta(q, \texttt{true})$, $\delta(q, \gguard)$ and $\delta(q, \lguard)$ are not defined.} 

	\textcolor{blue}{For the sake of convenience, from now on, we will use $\texttt{true}$ to refer to both guards $\texttt{true}$ and $n < N$, $\gguard$ to refer to both $\gguard$ and $\gaguard$, and $\lguard$ to refer to both $\lguard$ and $\laguard$.}

	\item \textbf{Output Distinction:} For any state $q\in Q$, if $\delta(q, \gguard) = (q_1, o_1, b_1, i_1)$ and $\delta(q, \lguard) = (q_2, o_2, b_2, i_2)$, then $o_1\neq o_2$ and at least one of $o_1\in \Gamma$ and $o_2\in \Gamma$ is true. \textcolor{blue}{In addition, $o_1\neq \phi$ and $o_2\neq \phi$ and if $\delta(q, n\geq N) = (q', o', b', i')$, then $o' = \phi$, i.e., the $\phi$ output symbol is reserved for transitions with guard $n\geq N$, which must output $\phi$}.

	\item \textbf{Initialization:} The initial state $q_0$ has only one outgoing transition of the form $\delta(q_0, \texttt{true}) = (q, o, \texttt{true}, i)$ for $i\in \{0, 1\}$.

	\item \textbf{Non-input transition:} From any $q\in Q_{non}$, if $\delta(q, c)$ is defined, then $c=\texttt{true}$.

	\item \textcolor{blue}{\textbf{Control Flow Separation:} Consider the underlying graph $G$ of $\mathcal{A}$. For all states $q\in Q$, if $\delta(q, n\geq N) = (q', o, b, i)$,$q$ and $q'$ must be in different strongly connected components of $G$}.

\end{itemize}

Note that the \textbf{control flow separation} condition implies that no cycle in $G$ can contain an edge that corresponds to a transition with guard $n\geq N$. In addition, determinism combined with control flow separation imply that no two transitions (i.e.\ transitions with different guards) can be from some state $q$ to the same state $q'$. 

\subsubsection{Path Probabilities}


\begin{defn} (from~\cite{chadhaLinearTimeDecidability2021})
	A \textbf{path} $\rho$ of length $n$ of a DiPA* $\mathcal{A}$ is a sequence of states, inputs, and outputs $\rho = q_0\xrightarrow{a_0, o_0} q_1\to\cdots \to q_{n-1}$, where $q_i$ are the states traversed in $\mathcal{A}$, $a_i$ are the inputs read in each state $q_i$, and $o_i$ are the outputs output by $\mathcal{A}$ at the transition $q_i\to q_{i+1}$. 
	We denote the sequence of inputs $a_i$ for a path $\rho$ as $\texttt{inseq}(\rho)$ and the sequence of outputs $o_i$ as $\texttt{outseq}(\rho)$. In general, for a path $\rho = q_0\to q_1\to \cdots \to q_{n-1}$ we denote the transition $q_i\to q_{i+1}$ by $\texttt{trans}(q[i])$ and the guard of $\texttt{trans}(\rho[i])$ as $\texttt{guard}(\rho[i])$.
\end{defn}

\begin{defn} (from~\cite{chadhaLinearTimeDecidability2021})
	Two paths $\rho = q_0\xrightarrow{a_0, o_0} q_1\to\cdots \to q_n$ and $\rho' = q'_0 \xrightarrow{a'_0, o'_0} q'_1 \to \cdots \to q'_n$ of a DiPA* $\mathcal{A}$ are \textbf{equivalent} if for all $i$, $o_i = o'_i$ and $q_i = o'_i$. In other words, $\rho$ and $\rho'$ traverse the same states in $\mathcal{A}$ and produce the same output, and only possibly differ in the inputs they read.
	(Note that due to output determinism, $\rho$ and $\rho'$ traverse the same states iff they produce the same output.)
\end{defn}

For any path $\rho$ of a DiPA* $\mathcal{A}$, we define $\PP[\epsilon, N, x, n, \rho]$ as the \textbf{probability} of path $\rho$ being traversed with $\mathcal{A}$ parameters $\epsilon$ and $N$, stored value $x$, and counter value $n$. This will enable us to define what it means for a DiPA* to be differentially private.

Consider a path $\rho = q_0\xrightarrow[]{a_0, o_0}q_1\rightarrow\ldots\xrightarrow[]{a_{n-1}, o_{n-1}} q_n$. Here, $a_i$ and $o_i$ are the input to state $q_i$ and output of transition $q_i\to q_{i+1}$, respectively (if $q_i\in Q_{non}$ i.e. $q$ does not take in input, $a_i = 0$).

If $|\rho| = 0$, we define $\PP[\epsilon, N, x, n, \rho] = 1$.
Otherwise, we define $\PP[\epsilon, N, x, n, \rho]$ recursively:
Let $P(q_0) = (d, \mu, d', \mu')$ be the parameters for sampling from Laplace distributions for $\texttt{insample}$ and $\texttt{insample}'$ at state $q_0$. Let $(q_0, c, q_1, o_0, b, i)$ represent the 0th transition, where $c$ is the guard of the 0th transition, $b$ is whether or not the 0th transition is an assignment transition, and $i$ is the amount that the counter $n$ gets incremented by in the 0th transition.

Let $\nu = \mu + a_0$. If $o_0 = (y, v, w)$ for $y\in \{\texttt{insample}, \texttt{insample'}\}$, then let \begin{align*}
	k &= \int_{v}^{w}\frac{d\epsilon}{2}e^{-d\epsilon|z-\mu-a_0|}dz\\
	k' &= \int_v^w\frac{d'\epsilon}{2}e^{-d'\epsilon|z-\mu'-a_0|}dz
\end{align*}


If the 0th transition of $\rho$ is not an assignment transition (i.e. $b = \texttt{false}$), then we define $\PP[\epsilon, N, x, n, \rho]$ as follows:

\textbf{Case 1:} $n\geq N$ and $c = n \geq N$.
If $o_0 \in \Gamma$, then $\PP[\epsilon, N, x, n, \rho] = \PP[\epsilon, N, x, n+i, \texttt{tail}(\rho)]$. If $o_0 = (\texttt{insample}, v, w)$ then $\PP[\epsilon, N, x, n, \rho] = k\PP[\epsilon, N, x+i, \texttt{tail}(\rho)]$. If $o_0 = (\texttt{insample}, v, w)$ then $\PP[\epsilon, N, x, n, \rho] = k'\PP[\epsilon, N, x, n+i, \texttt{tail}(\rho)]$

\textbf{Case 2:} $n < N$ and $c = n\geq N$. Then we define $\PP[\epsilon, N, x, n, \rho] = 0$.

Every case for other guards is exactly analogous to their counterpart definitions in~\cite{chadhaLinearTimeDecidability2021}, but in general where $\PP[\epsilon,N, x, n, \texttt{tail}(\rho)]$ is referenced in $\cite{chadhaLinearTimeDecidability2021}$, $\PP[\epsilon, N,x, n+i ,\texttt{tail}(\rho)]$ should be used instead.

Because of the initialization condition, for paths starting at the start state of $\mathcal{A}$, the starting value of $x$ is irrelevant. In addition, since $n$ is always initialized to 0, we will abuse notation for paths $\rho$ that start at the start state to write $\PP[\epsilon, N, \rho]$ to represent $\PP[\epsilon, N, x, 0, \rho]$.

We can use this definition of path probabilities to formalize what it means for paths to be valid program traces in $\mathcal{A}$:
\begin{defn}
	A path $\rho = q_0\to q_1\to\cdots q_n$ from the start state $q_0$ of $\mathcal{A}$ is \textbf{valid} if $\PP[\epsilon, N, \rho]> 0$.
\end{defn}


Most notably, given a definition of path probabilities, we can define what it means for a DiPA* to be differentially private:
\begin{defn}
	As in $\cite{chadhaLinearTimeDecidability2021}$, a DiPA* $\mathcal{A}$ with parameters $\epsilon, N$ is \textbf{$d\epsilon-$differentially private} if for all equivalent paths $\rho, \rho'$ in $\mathcal{A}$ such that $\texttt{inseq}(\rho)$ and $\texttt{inseq}(\rho')$ are adjacent, $\PP[\epsilon, N, \rho] \leq e^{d\epsilon}\PP[\epsilon, N, \rho']$.
\end{defn}

\subsubsection{Well-formedness}

In this section, we define what it means for a DiPA* to be \textbf{well-formed}, analogously to how the well-formedness of DiPAs are defined.


\begin{defn}
A \textbf{bounded} cycle $C$ in a DiPA* $\mathcal{A}$ is a cycle in $\mathcal{A}$ where there exists at least one transition $(q', \sigma, t, 1)$ (i.e. $\texttt{n}$ gets incremented) and there exists some $q\in Q$ (``exit state'') in the cycle such that $f(q, n \geq N) = (q', \sigma, t, i)$ where $q'$ is not in the cycle. Otherwise, the cycle is \textbf{unbounded}. 
\end{defn}

\begin{defn}
A cycle $C$ with an exit state with transition $n\geq N$ is an \textbf{infeasible} cycle if, for \textit{all} paths $\rho = q_0\to q_1\to \cdots \to q_m$ from the start state to a state $q_m \in C$, at least $N$ transitions $q_i \to q_{i+1}$ are increment transitions or some transition $q_i \to q_{i+1}$ has guard $n\geq N$. Otherwise, $C$ is \textbf{feasible}.
\end{defn}

\begin{defn} (from~\cite{chadhaLinearTimeDecidability2021}) A \textbf{leaking cycle} is a cycle $C = q_0\xrightarrow{a_0, o_0}q_1\to \cdots\to q_{n-1}\to q_0$ in a DiPA $\mathcal{A}$ if there exist indices $0\leq i< j<n$ such that the $i$th transition $q_i\to q_{i+1}$ is an assignment transition and the guard of the $j$th transition guard is not $n < N$ or $\texttt{true}$. 
\end{defn}

\begin{defn} (from~\cite{chadhaLinearTimeDecidability2021}) A cycle $\rho$ of a DiPA* $\mathcal{A}$ is an $\texttt{L}$-cycle (respectively, $\texttt{G}$-cycle) if there is an $i < |\rho|$ such that $\texttt{guard}(\rho[i]) = \lguard$ (respectively $\texttt{guard}(\rho[i]) = \gguard$).
\end{defn}

\begin{defn} (from~\cite{chadhaLinearTimeDecidability2021}) A path $\rho$ of a DiPA $\mathcal{A}^*$ is an $\texttt{AL}$-path (respectively, $\texttt{AG}$-path) if all assignment transitions on $\rho$ have guard $\lguard$ (respectively, $\gguard$).
\end{defn}

\begin{defn} (from~\cite{chadhaLinearTimeDecidability2021})	
	A pair of cycles $(C, C')$ in a DiPA $\mathcal{A}$ is a \textbf{leaking pair} if one of the following is satisfied:
	\begin{itemize}
		\item $C$ is an $\texttt{L}$-cycle, $C'$ is a $\texttt{G}$-cycle, and there is an $\texttt{AG}$-path from a state in $C$ to a state in $C'$.
		\item $C$ is an $\texttt{G}$-cycle, $C'$ is a $\texttt{L}$-cycle, and there is an $\texttt{AL}$-path from a state in $C$ to a state in $C'$.
	\end{itemize}
\end{defn}

\begin{defn}
	A pair of cycles $(C, C')$ is a \textbf{feasible unbounded leaking pair} of cycles in a DiPA* $\mathcal{A}$ if both $C$ and $C'$ are feasible and unbounded cycles, $C$ is an $\texttt{L}$-cycle (respectively, $\texttt{G}$-cycle), $C'$ is a $\texttt{G}$-cycle (respectively $\texttt{L}$-cycle), and there exists an $\texttt{AL}$-path (respectively, $\texttt{AG}$-path) $\rho = a_1 a_2\cdots a_k$ from $C$ to $C'$ (i.e.\ such that $a_1\in C$ and $a_k \in C'$) such that all of the following hold:
	\begin{enumerate}
		\item Either there are no $n \geq N$ transitions on $\rho$ or $C'$ has no exit state.
		\item Either there exists some path $\tau$ from the start state $q_0$ of $\mathcal{A}$ to $a_k$ that includes $a_1$ such that there are at most $N-1$ increment transitions on $\tau$ or $C'$ has no exit state.
		\item Either $C'$ has no exit state or $C$ has no increment transitions.
		\item If there exists an $n\geq N$ transition in $\rho$ from states $a_i$ to $a_{i+1}$, there exists some path $\tau$ from the start state $q_0$ of $\mathcal{A}$ to $a_i$ that includes $a_1$ such that there are at least $N$ increment transitions in $\tau$.
	\end{enumerate}

	Conditions (1)-(3) ensure that there exist some path in $\mathcal{A}$ such that either $n < N$ when entering $C'$ or that $C'$ has no exit state; otherwise, $C'$ would be rendered infeasible in practice.
	
	Condition (4) ensures that the path $\rho$ between $C$ and $C'$ is in fact traversible. 

	
\end{defn}


\begin{defn} (from~\cite{chadhaLinearTimeDecidability2021}) A cycle $C$ of a DiPA $\mathcal{A}$ is a \textbf{disclosing cycle} if there exists some $0\leq i <|C|$ such that $\texttt{trans}(C[i])$ is an input transition that outputs either $\texttt{insample}$ or $\texttt{insample}'$.
\end{defn}


\begin{defn} (adapted from~\cite{chadhaLinearTimeDecidability2021})
	An \textbf{feasible unbounded privacy violating lasso} is a path $\rho = a_1\to a_2\to \cdots\to a_k$ of length $n$ in a DiPA* $\mathcal{A}$ such that one of the following hold:

	\begin{itemize}
		\item $\texttt{tail}(\rho)$ is an $\texttt{AG}$-path (respectively, $\texttt{AL}$-path) such that $\texttt{last}(\rho)$ is in a
		\item $\rho$ is an $\texttt{AG}$-path (respectively, $\texttt{AL}$-path) such that $\texttt{first}(\rho)$ is in a feasible unbounded $\texttt{G}$-cycle (respectively, $\texttt{L}$-cycle) and the 0th transition has guard $\lguard$ (respectively, $\gguard$) and outputs $\texttt{insample}$ 
		\item $\rho$ is an $\texttt{AG}$-path (respectively, $\texttt{AL}$-path) such that $\texttt{first}(\rho)$ is in a feasible unbounded $\texttt{L}$-cycle (respectively, $\texttt{G}$-cycle) and the last transition has guard $\gguard$ (respectively, $\lguard$) and outputs $\texttt{insample}$.
	\end{itemize}

	In addition, if there are any transitions $a_i\to a_{i+1}$ in $\rho$ with guard $n\geq N$, there must exist some path represented by the word $\tau = \alpha \cdot \beta$ from the start state of $\mathcal{A}$ to $a_i$ such that $\alpha$ represents a path from the start state of $\mathcal{A}$ to $a_1$ and $\beta$ represents a subpath of $\rho$ from $a_1$ to $a_i$.

\end{defn}

\begin{defn}
For a lasso $\rho$, let $C_\rho$ be the cycle associated with\footnote{Hopefully this is clear} $\rho$. Then a lasso $\rho$ in a DiPA* $\mathcal{A}$ is bounded iff $C_\rho$ is bounded. Similarly, $\rho$ is feasible iff $C_\rho$ is feasible.
\end{defn}



\begin{defn} 
	A DiPA* $\mathcal{A}$ is well-formed if $\mathcal{A}$ has no reachable unbounded feasible leaking cycles, unbounded feasible leaking pair $(C, C')$ where $C$ is reachable, reachable unbounded feasible disclosing cycles, or reachable unbounded feasible privacy violating lassos.
\end{defn}




\subsection{Proving Differential Privacy}

\begin{thm}\label{DPtheorem}
	A DiPA* is well-formed if and only if it is differentially private.
\end{thm}


\begin{proof}

We prove Theorem~\ref{DPtheorem} by reducing a DiPA* to a DiPA and showing that they are equivalent through a construction similar to the canonical subset/powerset construction for finite automata.

Let $\mathcal{A}^* = (Q, \Sigma, \Gamma, q_0, X^*, P^*, \delta^*)$ be a well-formed DiPA* with parameters $\epsilon$ and $N$. 

Let $G = \{\texttt{true}, n<N, n\geq N, \gguard, \gaguard, \lguard,\laguard \}$ be the set of guard conditions for DiPA*s.

Construct the DiPA $\mathcal{A} = (Q \times [N], \Sigma, \Gamma\cup \{\phi\}, (q_0, 0), X, P, \delta)$ as follows:

For each state $q \in Q^*$:

For $g\in G$, if $\delta^*(q, g) = (q', \sigma, \mathbf{b}, x)$ is defined, define $\delta((q, k), g)$ as follows:

\textbf{Case 1: $g \in \{\texttt{true}, \gguard, \lguard\}$}

For all $k \in [N-1]$, define the transitions

\[
	\delta((q, k), g) = ((q', k+x), \sigma, \mathbf{b})	
\]

and define the transition
\[
	\delta((q, N), g) = ((q', N), \sigma, \mathbf{b})	
\]

\textbf{Case 2: $g = n \geq N$}

We define the transition 
\[
	\delta((q, N), \texttt{true}) = ((q', N), \sigma, \mathbf{b})
\]

\textbf{Case 3: $g = n < N$}

For all $k \in [N-1]$, define the transitions
\[
	\delta((q, k), \texttt{true}) = ((q', k+x), \sigma, \mathbf{b})	
\]

\textbf{Case 4: $g = \laguard $}

For all $k \in [N-1]$, define the transitions
\[
	\delta((q, k), \lguard) = ((q', k+x), \sigma, \mathbf{b})	
\]

\textbf{Case 5: $g = \gaguard $}

For all $k \in [N-1]$, define the transitions
\[
	\delta((q, k), \gguard) = ((q', k+x), \sigma, \mathbf{b})	
\]

Intuitively, at state $(q, k)$ in $\mathcal{A}$, $k$ will track the value of $n$ in $\mathcal{A}^*$ (since everything above $N$ is treated the same, we compress all of those values together).

For each state $(q, k)\in Q$, let $P((q, k)) = P^*(q)$.

\begin{clm}
	$\mathcal{A}$ is a valid DiPA.\@
\end{clm}
\begin{proof}
	To be a valid DiPA, $\delta$ must satisfy four conditions: determinism, output distinction, initialization, and non-input transition. 

	\textbf{Determinism:} Consider some state $(q, k)\in Q$ and suppose that $\delta((q, k), \texttt{true})$ is defined. Then either $\delta^*(q, \texttt{true})$ is defined, $\delta^*(q, n <N)$ and $k < N$, or $\delta^*(q, n\geq N)$ is defined and $k = N$. By the condition of determinism for DiPA*s, if $\delta^*(q, \texttt{true})$, then no other transitions from $q$ in $\mathcal{A}^*$ are defined. Thus, neither $\delta((q, k), \lguard)$ or $\delta((q, k), \gguard)$ can be defined in $\mathcal{A}$.

	If $\delta^*(q, n < N)$ is defined and $k < N$, then similarly none of $\delta^*(q, \gaguard)$, $\delta^*(q, \laguard)$, $\delta^*(q, \texttt{true})$, $\delta^*(q, \lguard)$, or $\delta^*(q, \gguard)$ can be defined. Additionally, since $ n < N$, there is no additional transition from $q$ corresponding to an $n\geq N$ guard in $\mathcal{A}^*$, so neither $\delta((q, k), \lguard)$ or $\delta((q, k), \gguard)$ can be defined in $\mathcal{A}$.

	Similarly, if $\delta^*(q, n \geq N)$ is defined and $k = N$, no other transitions from $q$ can be defined in $\mathcal{A}$.

	\textbf{Output distinction:} This follows immediately from the output distinction condition of DiPA*s.

	\textbf{Initialization:} By the initialization condition of DiPA*s, the initial state $q_0$ has only one outgoing transition of the form $\delta(q_0, \texttt{true}) = (q, o, \texttt{true}, i)$ for $i\in \{0, 1\}$. Thus, there is only one transition out of $(q_0, 0)$ in $\mathcal{A}$, with guard $\texttt{true}$.

	\textbf{Non-input transition:} Since input transitions are preserved from $\mathcal{A}^*$ in the construction of $\mathcal{A}$, this follows immediately from the condition of non-input transition for DiPA*s.
\end{proof}


\begin{lemma}\label{cyclelemma}
	If there exists a reachable cycle $C = (a_0, k_0)\to (a_1, k_1)\to \cdots \to (a_{m-1}, k_{m-1})\to (a_0, k_0)$ in $\mathcal{A}$ if and only if there exists a reachable unbounded feasible cycle $C^* = a_0\to a_1\to \cdots a_{m-1}\to a_0$ in $\mathcal{A}^*$.
\end{lemma}
\begin{proof}

	Let $C = (a_0, k_0)(a_1, k_1)\cdots (a_{m-1}, k_{m-1})(a_0, k_0)$ be a cycle in $\mathcal{A}$. Note that by construction, there must exist a cycle $C^*=a_0a_1\cdots a_{m-1}a_0$ in $\mathcal{A}^*$.

	Additionally note that for any path $(q, k) \to (q', k')$ in $\mathcal{A}$, $k' \geq k$. This implies that $k_0 = k_1 = \cdots = k_{m-1}$. 

	In order for $C^*$ to be bounded, there must be an increment transition between some states $a_i$ and $a_{i+1}$ in $C^*$. However, this would mean that there exists a transition  $(a_i, k_i)\to (a_{i+1}, k_i + 1)$, which is impossible because all $k_i$'s are equal for $ 0\leq i \leq m-1$ and because $\mathcal{A}$ is deterministic. Therefore $C^*$ is unbounded.

	Consider the underlying graphs $G_\mathcal{A}, G_{\mathcal{A}^*}$ of $\mathcal{A}$ and $\mathcal{A}^*$, respectively. Consider some edge $e^* = (q, q')\in G_{\mathcal{A}^*}$; there exists exactly one corresponding edge $e = ((q, k), (q', k'))$ in $G_{\mathcal{A}}$.

	Suppose that $C^*$ is infeasible for the sake of contradiction. Then $k_0 = k_1 = \cdots k_{m-1} = n$ and there is some exit state $s\in C^*$ such that $\delta^*(s, n\geq N) = (s', \sigma, \mathbf{b}, x)$. Note that because of determinism, there is exactly one transition $(s, n)\to (s', n)$ in $G$ out of the state $(s, n)$. 
	Therefore, $(s', n)$ must also be in the cycle $C$, which implies $s'$ must be in the cycle $C^*$. However, this contradicts control flow separation, since then $s$ and $s'$ would be in the same component of $G_{\mathcal{A}^*}$, even with the edge corresponding to the $n\geq N$ transition removed. Thus, $C^*$ is feasible.

	Now let $C^* = a_0 \to \ldots \to a_m \to a_0$ be an reachable unbounded feasible cycle in $\mathcal{A}^*$. Because $C^*$ is feasible, there exists some path from the start state $q_0$ of $\mathcal{A}^*$ to $a_0$ such that $n < N$ at $a_0$. Let $k_0$ be the minimum such value of $n$ at $a_0$. Then consider the sequence of states $(a_0, k_0), (a_1, k_1), \ldots, (a_m, n_m), (a_0, k_0)$ in $\mathcal{A}$ such that $k_i$ is defined as follows:

	Let $(a_{i-1}, c, a_i, o, b, j)$ represent the transition from $a_{i-1}$ to $a_i$ in $\mathcal{A}^*$. Then for $i > 0, k_i = k_{i-1} + j$. Note that for all $1 \leq i\leq m$, there exists a transition $(a_{i-1}, k_{i-1}) \to (a_i, k_i)$ by construction.

	Suppose that there is no increment transition in $C^*$. Then $\forall 1\leq i \leq m, k_i = k_{i-1}$. Thus, $k_m = k_0$ and there exists a transition $(a_m, k_0)\to (a_0, k_0)$ in $\mathcal{A}$. Thus, the cycle $C = (a_0, k_0)\to (a_1, k_1)\to \ldots\to (a_m, k_m) \to (a_0, k_0)$ is a cycle in $\mathcal{A}$.

	Now suppose that $C^*$ has an increment transition but no exit state. Then the state $(a_0, N)$ is reachable in $\mathcal{A}$. To see this, consider the path $(a_0, k_0)\to (a_1, k_1)\to \ldots\to (a_m, k_m) \to (a_0, k'_0)$ in $\mathcal{A}$. Since $C^*$ has an increment transition, $k'_0 > k_0$. 
	Since $C^*$ has no exit state and $k_i$ is bounded above by $N$, this means that there exists a path from $(q_0, 0)$ to $(a_0, k_0)$ to $(a_0, N)$. Then again because $C^*$ has no exit state, $C = (a_0, N)\to (a_1, N)\to \ldots\to (a_m, N) \to (a_0, N)$ is a cycle in $\mathcal{A}$.
\end{proof}

\begin{lemma}\label{wellformedequiv}
	If $\mathcal{A}^*$ is well-formed if and only if $\mathcal{A}$ is well-formed.	
\end{lemma}
\begin{proof}

	By lemma~\ref{cyclelemma}, there exists a reachable cycle $C = (q_1, n_1)\to\ldots \to (q_m, n_m) \to (q_1, n_1)$ in $\mathcal{A}$ iff $C^* = q_1\to\ldots q_m\to q_1$ is a reachable unbounded feasible cycle in $\mathcal{A}^*$. Further, such a $C$ is a leaking cycle iff there exist indices $0\leq i < j < n$ such that the transition $(q_i, n_i)\to (q_{i+1}, n_{i+1})$ is an assignment transition and $\texttt{guard}((q_j, n_j))$ is not $\texttt{true}$. 
	By construction, this happens iff the transition $q_i\to q_{i+1}$ in $\mathcal{A}^*$ is also an assignment transition and $\texttt{guard}(q_j)$ in $\mathcal{A}^*$ is not $\texttt{true}$ or $n < N$, so $C^*$ is a leaking cycle. Thus, there exists a reachable leaking cycle in $\mathcal{A}$ iff there exists a reachable unbounded feasible leaking cycle in $\mathcal{A}^*$.

	Suppose that $C = (q_1, n_1)\to\ldots \to (q_m, n_m) \to (q_1, n_1)$ and $C' = (q'_1, n'_1)\to\ldots \to (q'_m, n'_m) \to (q'_1, n'_1)$ are cycles in $\mathcal{A}$ such that $C$ is reachable and $(C, C')$ is a leaking pair in $\mathcal{A}$. By lemma~\ref{cyclelemma}, $C^* = q_1\to\ldots q_m\to q_1$ and $C^{\prime *} = q'_1 \to \ldots q'_m \to q'_1$ are unbounded and feasible cycles in $\mathcal{A}^*$ and $C^*$ is reachable.
	Without loss of generality, suppose that $C$ is an $\texttt{L}$-cycle and $C'$ is a $\texttt{G}$-cycle. Then by construction, $C^*$ is an $\texttt{L}$-cycle and $C^{\prime *}$ is a $\texttt{G}$-cycle. In addition, there exists some $\texttt{AL}-$path $\rho = (a_1, k_1)\to\ldots \to (a_l, k_l)$ in $\mathcal{A}$ from $C$ to $C'$. By construction, there must exist some $\texttt{AL}-$path $\rho^* = a_1\to\ldots\to a_l$ in $\mathcal{A}^*$ from $C^*$ to $C^{\prime *}$. 

	Note that if there exists a transition with guard $n \geq N$ in $\rho^*$, then $k_l = n$. 

	
	Observe that if $(q, k)\to (q', k')$ is an assignment transition in $\mathcal{A}$, $q \to q'$ is also an assignment transition in $\mathcal{A}^*$. 
	Similarly, if a transition $(q, k)\to (q', k')$ has a guard of $\gguard$ (respectively, $\lguard$) in $\mathcal{A}$, $q\to q'$ also has a guard of $\gguard$ (respectively, $\lguard$) in $\mathcal{A}^*$. Together, these mean that leaking cycles, leaking pairs, disclosing cycles, and privacy violating lassos in $\mathcal{A}$ correspond to their feasible unbounded equivalents in $\mathcal{A}^*$.

\end{proof}

\begin{lemma}\label{pathequivalence}
	Let $\Psi = \{\rho: \rho$ is a path in $\mathcal{A}\}$ and $\Psi^* = \{\rho^*: \rho^*$ is a valid path in $\mathcal{A}^*\}$ be the sets of paths in $\mathcal{A}$ and $\mathcal{A}^*$, respectively. There exists a bijection $f: \Psi\to \Psi^*\times [N]$ such that $\forall x, \forall \rho\in \Psi,$ if $f(\rho) = (\rho^*, n), \PP[\epsilon, x, \rho] = \PP[\epsilon, N, x, n, \rho^*]$.
\end{lemma}

\begin{proof}
	Let $\rho = (q_1, n_1)\to(q_2, n_2)\to \ldots \to (q_m, n_m)$ be a path in $\mathcal{A}$. Then let $f(\rho) = (q_1\to q_2\ldots \to q_m, n_1)$ such that $\texttt{inseq}(q_1\to q_2\ldots \to q_m) = \texttt{inseq}(\rho)$. Note that $\texttt{outseq}(q_1\to q_2\ldots \to q_m) = \texttt{outseq}(\rho)$ by output determinism.
	
	By construction\footnote{Does this need to be elaborated on?}, $\rho^* = q_1\to\ldots\to q_m$ must be a valid path in $\mathcal{A}^*$ if the value of the variable $n$ in $\mathcal{A}^*$ is $n_1$ at $q_1$.

	$f$ is injective: Let $\rho = (q_1, n_1)\to \ldots \to (q_m, n_m), \rho' = (q'_1, n'_1)\to \ldots \to (q'_m, n'_m)$ be two paths in $\mathcal{A}$ such that $\rho \neq \rho'$. If $|\rho| \neq |\rho'|, f(\rho) \neq f(\rho')$ clearly. 
	Suppose $|\rho| = |\rho'|$ and consider the smallest $i$ such that either $n_i \neq n'_i$ or $q_i \neq q'_i$. If $q_i \neq q'_i$, then clearly $f(\rho) = (q_1\to\ldots\to q_i\to \ldots \to q_m, n_1)\neq (q_1\to\ldots\to q'_i\to \ldots \to q_m, n'_1) = f(\rho')$. 
	Otherwise, if $q_i = q'_i$ and $n_i \neq n'_i$, note that $i = 1$: there can only be one transition from $q_{i-1}\to q_{i}$ in $\mathcal{A}^*$ and because $i$ is the smallest such $i$, $q_{i-1} = q'_{i-1}$. Thus if $i > 1$, this would mean that $n_i = n'_i$, which is impossible. So $f(\rho) = (q_1\to\ldots\to q_m, n_1) \neq (q'_1\to\ldots\to q'_m, n'_1) = f(\rho')$.

	$f$ is surjective: Let $(\rho^*, n_1) = (q_1\to\ldots \to q_m, n_1) \in \Psi^* \times [N]$. Let $n_i$ be the value of $n$ in $\mathcal{A}^*$ after starting at state $q_1$ with $n=n_1$ and traversing each state $q_i$ in order. Then $\rho = (q_1, n_1)\to \ldots (q_m, n_m)$ is a path in $\mathcal{A}$ by construction and clearly $f(\rho) = (\rho^*, n_1)$.

	Fix $x\in \RR$ and $\rho \in \Psi$. Let $f(\rho) = (\rho^*, n_1)$. We will show that $\PP[\epsilon, x, \rho] = \PP[\epsilon, N, x, n_1, \rho^*]$.

	This follows by induction on $|\rho^*|$:

	Suppose $|\rho^*| = 0$. Then $\PP[\epsilon, x_0, \rho] = \PP[\epsilon, N, x_0, n_0, \rho^*] = 1$.

	Now suppose $|\rho^*| = k > 0$ and that for all $|\rho'^{*}|<k$, $\PP[\epsilon, x_0, \rho'] = \PP[\epsilon, N, x_0, n_0, \rho'^{*}]$.

	Let $c_0$ be the guard of the first transition $q_0 \to q_1$ in $\rho^* = q_0q_1\cdots q_{m-1}$. So $\delta^*(q_0, c_0) = (q_1, \sigma, b, i)$.

		
	Let $\nu = \mu + a_0$, where $a_0$ is the first input value read (or 0 if $q_0\in Q_{non}$). Let $P(q_0)= (d, \mu, d', \mu')$ be the parameters for sampling from the Laplace distribution at $q_0$. 
	Define $\ell$ and $u$ as follows: if $o_0 \in \Gamma\cup \{\phi\}$, then $\ell = -\infty$ and $u = \infty$. Otherwise, if $o_0 = (y, v, w)$ for $y \in \{\texttt{insample}, \texttt{insample'}\}$ (i.e. either $\texttt{insample}$ or $\texttt{insample}'$ is output with a value between $v$ and $w$), then $\ell = v$ and $u = w$.
	
	If $o_0 = (y, v, w)$ for $y\in \{\texttt{insample}, \texttt{insample'}\}$, then let \begin{align*}
		k &= \int_{v}^{w}\frac{d\epsilon}{2}e^{-d\epsilon|z-\mu-a_0|}dz\\
		k' &= \int_v^w\frac{d'\epsilon}{2}e^{-d'\epsilon|z-\mu'-a_0|}dz
	\end{align*}

	\textbf{Case 1: $c = n < N$}

	Note that $ n_0 < N$ since $\rho^*$ is a valid path.
	
	By construction, $\delta((q_0, n_0), \texttt{true}) = ((q_1, n_1), \sigma, b)$ where $n_1 = n_0 + i$. Let $x'$ be the value of $x$ at $q_1$ in $\mathcal{A}^*$. Since $\mathcal{A}^*$ assigns to $x$ iff $\mathcal{A}$ does, $x'$ is also the value of $x$ at $(q_1, n_1)$ in $\mathcal{A}$.

	Since $n_0 < N$, by the induction hypothesis
		\[
			\PP[\epsilon, N, x, n_0, \rho^*] = \PP[\epsilon, N, x', n_1, \texttt{tail}(\rho^*)] = \PP[\epsilon, x', \texttt{tail}(\rho)] = \PP[\epsilon, x, \rho]
		\]

	\textbf{Case 2: $c = \texttt{true}$}

	As in case 1, by construction, $\delta((q_0, n_0), \texttt{true}) = ((q_1, n_1), \sigma, b)$ where $n_1 = n_0 + i$. Let $x'$ be the value of $x$ at $q_1$ in $\mathcal{A}^*$. Since $\mathcal{A}^*$ assigns to $x$ iff $\mathcal{A}$ does, $x'$ is also the value of $x$ at $(q_1, n_1)$ in $\mathcal{A}$.

	Then by the induction hypothesis
		\[
			\PP[\epsilon, N, x, n_0, \rho^*] = \PP[\epsilon, N, x', n_1, \texttt{tail}(\rho^*)] = \PP[\epsilon, x', \texttt{tail}(\rho)] = \PP[\epsilon, x, \rho]
		\]

	\textbf{Case 3: $c = n\geq N$}

	Note that $ n_0 \geq N$ since $\rho^*$ is a valid path.
	
	By construction, $\delta((q_0, n_0), \texttt{true}) = ((q_1, n_1), \sigma, b)$ where $n_1 = n_0$. Let $x'$ be the value of $x$ at $q_1$ in $\mathcal{A}^*$. Since $\mathcal{A}^*$ assigns to $x$ iff $\mathcal{A}$ does, $x'$ is also the value of $x$ at $(q_1, n_1)$ in $\mathcal{A}$.

	Since $n_0 \geq N$, by the induction hypothesis
		\[
			\PP[\epsilon, N, x, n_0, \rho^*] = \PP[\epsilon, N, x', n_1, \texttt{tail}(\rho^*)] = \PP[\epsilon, x', \texttt{tail}(\rho)] = \PP[\epsilon, x, \rho]
		\]

	\textbf{Case 4: $c = \gaguard$}

	Since $\rho^*$ is valid, $n_0 < N$. 

	By construction, $\delta((q_0, n_0), \gguard) = ((q_1, n_1), \sigma, b)$ where $n_1 = n_0+i$. 

	Suppose $b = \texttt{true}$ (i.e. $\texttt{trans}(q_0)$ is an assignment transition), then:

	If $\sigma$ is of the form $(\texttt{insample'}, v, w)$, since $n_0 < N$, 
		\begin{align*}
			\PP[\epsilon, N, x, n_0, \rho^*] &= k'\left(\int_x^\infty\frac{d\epsilon}{2}e^{-d\epsilon|z-\nu|}\right)\PP[\epsilon, N, z, n_1, \texttt{tail}(\rho^*)]dz \\
			&= k'\left(\int_x^\infty\frac{d\epsilon}{2}e^{-d\epsilon|z-\nu|}\right)\PP[\epsilon, z, \texttt{tail}(\rho)]dz \text{ by the induction hypothesis }\\
			&= \PP[\epsilon, x, \rho]
		\end{align*}
	
	Otherwise, 
	\begin{align*}
		\PP[\epsilon, N, x, n_0, \rho^*] &= \left(\int_{\max(x, \ell)}^u\frac{d\epsilon}{2}e^{-d\epsilon|z-\nu|}\right)\PP[\epsilon, N, z, n_1, \texttt{tail}(\rho^*)]dz \\
		&= \left(\int_{\max(x, \ell)}^u\frac{d\epsilon}{2}e^{-d\epsilon|z-\nu|}\right)\PP[\epsilon, z, \texttt{tail}(\rho)]dz \text{ by the induction hypothesis }\\
		&= \PP[\epsilon, x, \rho]
	\end{align*}

	Now suppose that $b = \texttt{false}$. If $\sigma$ is of the form $(\texttt{insample'}, v, w)$, since $n_0 < N$, 
	\begin{align*}
		\PP[\epsilon, N, x, n_0, \rho^*] &= k'\left(\int_x^\infty\frac{d\epsilon}{2}e^{-d\epsilon|z-\nu|}dz\right)\PP[\epsilon, N, x, n_1, \texttt{tail}(\rho^*)]\\
		&= k'\left(\int_x^\infty\frac{d\epsilon}{2}e^{-d\epsilon|z-\nu|}dz\right)\PP[\epsilon, x, \texttt{tail}(\rho)] \text{ by the induction hypothesis }\\
		&= \PP[\epsilon, x, \rho]
	\end{align*}

	Otherwise, 
	\begin{align*}
		\PP[\epsilon, N, x, n_0, \rho^*] &= \left(\int_{\max(x, \ell)}^u\frac{d\epsilon}{2}e^{-d\epsilon|z-\nu|}dz\right)\PP[\epsilon, N, x, n_1, \texttt{tail}(\rho^*)]dz \\
		&= \left(\int_{\max(x, \ell)}^u\frac{d\epsilon}{2}e^{-d\epsilon|z-\nu|}dz\right)\PP[\epsilon, x, \texttt{tail}(\rho)] \text{ by the induction hypothesis }\\
		&= \PP[\epsilon, x, \rho]
	\end{align*}

	\textbf{Case 5: $c = \laguard$}

	Since $\rho^*$ is valid, $n_0 < N$. 

	By construction, $\delta((q_0, n_0), \lguard) = ((q_1, n_1), \sigma, b)$ where $n_1 = n_0+i$. 

	Suppose $b = \texttt{true}$. Then if $\sigma$ is of the form $(\texttt{insample'}, v, w)$, since $n_0 < N$, 
		\begin{align*}
			\PP[\epsilon, N, x, n_0, \rho^*] &= k'\left(\int_{-\infty}^x\frac{d\epsilon}{2}e^{-d\epsilon|z-\nu|}\right)\PP[\epsilon, N, z, n_1, \texttt{tail}(\rho^*)]dz \\
			&= k'\left(\int_{-\infty}^x\frac{d\epsilon}{2}e^{-d\epsilon|z-\nu|}\right)\PP[\epsilon, z, \texttt{tail}(\rho)]dz \text{ by the induction hypothesis }\\
			&= \PP[\epsilon, x, \rho]
		\end{align*}

	
	Otherwise, 
	\begin{align*}
		\PP[\epsilon, N, x, n_0, \rho^*] &= \left(\int_{\ell}^{\min(u, x)}\frac{d\epsilon}{2}e^{-d\epsilon|z-\nu|}\right)\PP[\epsilon, N, z, n_1, \texttt{tail}(\rho^*)]dz \\
		&= \left(\int_{\ell}^{\min(u, x)}\frac{d\epsilon}{2}e^{-d\epsilon|z-\nu|}\right)\PP[\epsilon, z, \texttt{tail}(\rho)]dz \text{ by the induction hypothesis }\\
		&= \PP[\epsilon, x, \rho]
	\end{align*}

	Now suppose that $b = \texttt{false}$. If $\sigma$ is of the form $(\texttt{insample'}, v, w)$, since $n_0 < N$, 
	\begin{align*}
		\PP[\epsilon, N, x, n_0, \rho^*] &= k'\left(\int_x^\infty\frac{d\epsilon}{2}e^{-d\epsilon|z-\nu|}dz\right)\PP[\epsilon, N, x, n_1, \texttt{tail}(\rho^*)]\\
		&= k'\left(\int_x^\infty\frac{d\epsilon}{2}e^{-d\epsilon|z-\nu|}dz\right)\PP[\epsilon, x, \texttt{tail}(\rho)] \text{ by the induction hypothesis }\\
		&= \PP[\epsilon, x, \rho]
	\end{align*}

	Otherwise, 
	\begin{align*}
		\PP[\epsilon, N, x, n_0, \rho^*] &= \left(\int_{\ell}^{\min(u, x)}\frac{d\epsilon}{2}e^{-d\epsilon|z-\nu|}dz\right)\PP[\epsilon, N, x, n_1, \texttt{tail}(\rho^*)]dz \\
		&= \left(\int_{\ell}^{\min(u, x)}\frac{d\epsilon}{2}e^{-d\epsilon|z-\nu|}dz\right)\PP[\epsilon, x, \texttt{tail}(\rho)] \text{ by the induction hypothesis }\\
		&= \PP[\epsilon, x, \rho]
	\end{align*}


	\textbf{Case 6: $c = \gguard$}
	
	By construction, $\delta((q_0, n_0), \gguard) = ((q_1, n_1), \sigma, b)$ where $n_1 = n_0+i$. 

	Suppose $b = \texttt{true}$ (i.e. $\texttt{trans}(q_0)$ is an assignment transition), then:

	If $\sigma$ is of the form $(\texttt{insample'}, v, w)$, 
		\begin{align*}
			\PP[\epsilon, N, x, n_0, \rho^*] &= k'\left(\int_x^\infty\frac{d\epsilon}{2}e^{-d\epsilon|z-\nu|}\right)\PP[\epsilon, N, z, n_1, \texttt{tail}(\rho^*)]dz \\
			&= k'\left(\int_x^\infty\frac{d\epsilon}{2}e^{-d\epsilon|z-\nu|}\right)\PP[\epsilon, z, \texttt{tail}(\rho)]dz \text{ by the induction hypothesis }\\
			&= \PP[\epsilon, x, \rho]
		\end{align*}

	
	Otherwise, 
	\begin{align*}
		\PP[\epsilon, N, x, n_0, \rho^*] &= \left(\int_{\max(x, \ell)}^u\frac{d\epsilon}{2}e^{-d\epsilon|z-\nu|}\right)\PP[\epsilon, N, z, n_1, \texttt{tail}(\rho^*)]dz \\
		&= \left(\int_{\max(x, \ell)}^u\frac{d\epsilon}{2}e^{-d\epsilon|z-\nu|}\right)\PP[\epsilon, z, \texttt{tail}(\rho)]dz \text{ by the induction hypothesis }\\
		&= \PP[\epsilon, x, \rho]
	\end{align*}

	Now suppose that $b = \texttt{false}$. If $\sigma$ is of the form $(\texttt{insample'}, v, w)$, 
	\begin{align*}
		\PP[\epsilon, N, x, n_0, \rho^*] &= k'\left(\int_x^\infty\frac{d\epsilon}{2}e^{-d\epsilon|z-\nu|}dz\right)\PP[\epsilon, N, x, n_1, \texttt{tail}(\rho^*)]\\
		&= k'\left(\int_x^\infty\frac{d\epsilon}{2}e^{-d\epsilon|z-\nu|}dz\right)\PP[\epsilon, x, \texttt{tail}(\rho)] \text{ by the induction hypothesis }\\
		&= \PP[\epsilon, x, \rho]
	\end{align*}


	Otherwise, 
	\begin{align*}
		\PP[\epsilon, N, x, n_0, \rho^*] &= \left(\int_{\max(x, \ell)}^u\frac{d\epsilon}{2}e^{-d\epsilon|z-\nu|}dz\right)\PP[\epsilon, N, x, n_1, \texttt{tail}(\rho^*)]dz \\
		&= \left(\int_{\max(x, \ell)}^u\frac{d\epsilon}{2}e^{-d\epsilon|z-\nu|}dz\right)\PP[\epsilon, x, \texttt{tail}(\rho)] \text{ by the induction hypothesis }\\
		&= \PP[\epsilon, x, \rho]
	\end{align*}

	\textbf{Case 7: $c = \lguard$}

	By construction, $\delta((q_0, n_0), \lguard) = ((q_1, n_1), \sigma, b)$ where $n_1 = n_0+i$. 

	Suppose $b = \texttt{true}$. Then if $\sigma$ is of the form $(\texttt{insample'}, v, w)$, 
		\begin{align*}
			\PP[\epsilon, N, x, n_0, \rho^*] &= k'\left(\int_{-\infty}^x\frac{d\epsilon}{2}e^{-d\epsilon|z-\nu|}\right)\PP[\epsilon, N, z, n_1, \texttt{tail}(\rho^*)]dz \\
			&= k'\left(\int_{-\infty}^x\frac{d\epsilon}{2}e^{-d\epsilon|z-\nu|}\right)\PP[\epsilon, z, \texttt{tail}(\rho)]dz \text{ by the induction hypothesis }\\
			&= \PP[\epsilon, x, \rho]
		\end{align*}
	
	Otherwise, 
	\begin{align*}
		\PP[\epsilon, N, x, n_0, \rho^*] &= \left(\int_{\ell}^{\min(u, x)}\frac{d\epsilon}{2}e^{-d\epsilon|z-\nu|}\right)\PP[\epsilon, N, z, n_1, \texttt{tail}(\rho^*)]dz \\
		&= \left(\int_{\ell}^{\min(u, x)}\frac{d\epsilon}{2}e^{-d\epsilon|z-\nu|}\right)\PP[\epsilon, z, \texttt{tail}(\rho)]dz \text{ by the induction hypothesis }\\
		&= \PP[\epsilon, x, \rho]
	\end{align*}

	Now suppose that $b = \texttt{false}$. If $\sigma$ is of the form $(\texttt{insample'}, v, w)$, 
	\begin{align*}
		\PP[\epsilon, N, x, n_0, \rho^*] &= k'\left(\int_x^\infty\frac{d\epsilon}{2}e^{-d\epsilon|z-\nu|}dz\right)\PP[\epsilon, N, x, n_1, \texttt{tail}(\rho^*)]\\
		&= k'\left(\int_x^\infty\frac{d\epsilon}{2}e^{-d\epsilon|z-\nu|}dz\right)\PP[\epsilon, x, \texttt{tail}(\rho)] \text{ by the induction hypothesis }\\
		&= \PP[\epsilon, x, \rho]
	\end{align*}

	Otherwise, 
	\begin{align*}
		\PP[\epsilon, N, x, n_0, \rho^*] &= \left(\int_{\ell}^{\min(u, x)}\frac{d\epsilon}{2}e^{-d\epsilon|z-\nu|}dz\right)\PP[\epsilon, N, x, n_1, \texttt{tail}(\rho^*)]dz \\
		&= \left(\int_{\ell}^{\min(u, x)}\frac{d\epsilon}{2}e^{-d\epsilon|z-\nu|}dz\right)\PP[\epsilon, x, \texttt{tail}(\rho)] \text{ by the induction hypothesis }\\
		&= \PP[\epsilon, x, \rho]
	\end{align*}

	This is sufficient to prove the lemma. 
	
\end{proof}

\begin{lemma}\label{dpequiv}
	There exists $d > 0$ such that $\mathcal{A}$ is $d\epsilon-$differentially private if and only if $\mathcal{A}^*$ is $d\epsilon$-differentially private.
\end{lemma}

\begin{proof}
	Let $f$ be a bijection from paths in $\mathcal{A}$ to tuples of paths in $\mathcal{A}^*$ to $[N]$, as defined in Lemma \ref{pathequivalence}.

	Suppose that $\exists d > 0$ such that $\mathcal{A}$ is $d\epsilon-$differentially private. Then for all equivalent paths $\rho, \rho'$ in $\mathcal{A}$ from the start state $(q_0, 0)$ such that $\texttt{inseq}(\rho)$ and $\texttt{inseq}(\rho')$ are adjacent, $\PP[\epsilon, \rho] \leq e^{d\epsilon} \PP[\epsilon, \rho']$.
	Consider two equivalent paths $\rho^*, \rho^{\prime *}$ in $\mathcal{A}^*$ from the start state $q_0$ such that $\texttt{inseq}(\rho^*)$ and $\texttt{inseq}(\rho^{\prime *})$ are adjacent. Then $\forall x\in \RR, \PP[\epsilon, N, x, 0, \rho^*] = \PP[\epsilon, x, f^{-1}((\rho^*, 0))] \leq e^{d\epsilon}\PP[\epsilon, x, f^{-1}((\rho^{\prime *}, 0))] = e^{d\epsilon}\PP[\epsilon, N, x, 0, \rho^{\prime *}]$. Thus, $\mathcal{A}^*$ is $d\epsilon$-differentially private.

	Suppose that $\forall d>0, \mathcal{A}$ is not $d\epsilon$-differentially private. So there exists two equivalent paths from the start state $(q_0, 0)$ in $\mathcal{A}$ $\rho = (q_0, 0)\to\ldots\to (q_m, n_m), \rho' = (q_0, 0) \to (q'_m, n'_m)$ in $\mathcal{A}$ such that $\texttt{inseq}(\rho)$ and $\texttt{inseq}(\rho')$ are adjacent, but $\PP[\epsilon, \rho] > e^{d\epsilon}\PP[\epsilon, \rho']$. 
	
	Let $f(\rho) = (\rho^*, 0)$ and $f(\rho') = (\rho^{\prime *}, 0)$. Fix $\texttt{inseq}(\rho^*) = \texttt{inseq}(\rho)$ and $\texttt{inseq}(\rho^{\prime *})=\texttt{inseq}(\rho^{\prime})$.

	Then by Lemma \ref{pathequivalence}, $\PP[\epsilon, N, \rho]=\PP[\epsilon, N, x, 0, \rho^*] = \PP[\epsilon, x, \rho] = \PP[\epsilon, \rho] > e^{d\epsilon}\PP[\epsilon, \rho'] = e^{d\epsilon}\PP[\epsilon, x, \rho'] = e^{d\epsilon}\PP[\epsilon, N, x, 0, \rho^{\prime *}]= e^{d\epsilon}\PP[\epsilon, N, \rho^{\prime *}]$. Thus, $\mathcal{A}^*$ is not $d\epsilon$-DP.
\end{proof}

Lemmas~\ref{wellformedequiv} and~\ref{dpequiv} together prove the theorem. 
\end{proof}

\begin{cor}	

	Let $\mathcal{A}^*$ be a DiPA* with unfixed parameters.

	Let $f(\epsilon, N):\RR\times \NN \to \RR$ be defined as follows:

	Consider the instantiated version of $\mathcal{A}^*$ with parameters $\epsilon$ and $N$. Let $\mathcal{A}$ be the DiPA constructed from $\mathcal{A}^*$ as in Theorem 3.1. $f(\epsilon, N) = wt(\mathcal{A})$.
	
	Then $\forall \epsilon, f(\epsilon, N)$ grows linearly in $N$.
\end{cor}

\begin{cor}
	For a DiPA* $\mathcal{A}^*$, the well-formedness of $\mathcal{A}^*$ can be decided efficiently.
\end{cor}
\begin{proof}
	Note that the time it takes is bounded by the cost of creating a DiPA $\mathcal{A}$ from $\mathcal{A}^*$ as in Theorem 3.1. The construction of $\mathcal{A}$ from $\mathcal{A}^*$ causes the number of states to increase by a factor of $N$. 
	Each transition in $\mathcal{A}^*$ corresponds to at most $N$ transitions in $\mathcal{A}$. Since the well-formedness of $\mathcal{A}$ and $\mathcal{A}^*$ are equivalent, at most there is a linear increase in the time required to check the well-formedness of $\mathcal{A}^*$ as compared to a DiPA* of the same size.
\end{proof}

\newpage

\section{Input Terminator: EDIT I HAVE REALIZED THAT THIS IS BASICALLY MEANINGLESS}

\subsection{Definitions}

\begin{defn}
	Fix a parameter $\epsilon$. Let $C$ be the guard conditions $\{\texttt{input}=\tau, \texttt{input}\neq \tau, \itgguard, \itlguard\}$. An \textbf{Input Terminated Differentially Private Automaton} (ITDiPA) $\mathcal{A}$ is defined as the tuple $\mathcal{A} = (Q, \Sigma, \Gamma, q_0, X, P, \delta)$, where:
	\begin{itemize}
		\item $Q = $ finite set of states; partitioned into input states $Q_{in}$ and non-input states $Q_{non}$
		\item $\Sigma \cup \{\tau\}$ is the input alphabet ($\Sigma$ is taken to be $\RR$)
		\item $\Gamma$ is a finite output alphabet
		\item $q_0\in Q$ is the starting state
		\item $X = \{\texttt{x}, \texttt{insample}, \texttt{insample'}\}$ is a set of variables. $\texttt{x}, \texttt{insample}, \texttt{insample'} \in \RR$.
		\item $P: Q \to \QQ^{\geq 0} \times \QQ \times \QQ^{\geq 0} \times \QQ$ describing the parameters for sampling from Laplace distributions at each state.
		\item $\delta: (Q \times C) \to Q\times (\Gamma \cup \{\texttt{insample}, \texttt{insample'}\}) \times \{\texttt{true}, \texttt{false}\}$ is the transition function (technically a relation) that defines what state to transition to, what symbol or real value to output, and whether or not $x$ is assigned to.
	\end{itemize}
\end{defn}

There are certain conditions that $\delta$ must satisfy; these are almost all the same as the restrictions on transition functions of DiPA, but with some slight modifications (marked in \textcolor{blue}{blue}):
\begin{itemize}
	\item \textbf{Determinism:} 
	If, for a state $q\in Q$, a transition from $q$ with guard $\texttt{input}\neq \tau$ is defined, then there are no transitions from $q$ with guard either $\itlguard$ nor $\itgguard$.

	\textcolor{blue}{Note that if the automaton is in state $q$ and none of the guards of outgoing transitions from $q$ are satisfied, the automaton terminates.}
	
	\item \textbf{Output Distinction:} For any state $q\in Q$, if $\delta(q, \gguard) = (q_1, o_1, b_1)$ and $\delta(q, \lguard) = (q_2, o_2, b_2)$, then $o_1\neq o_2$ and at least one of $o_1\in \Gamma$ and $o_2\in \Gamma$ is true.

	\item \textbf{Initialization:} The initial state $q_0$ has only one outgoing transition of the form $\delta(q_0, \texttt{input}\neq\tau) = (q, o, \texttt{true})$.

	\item \textbf{Non-input transition:} From any $q\in Q_{non}$, if $\delta(q, c)$ is defined, then $c\in \{\texttt{input}\neq \tau, \texttt{input} = \tau\}$.

	\item \textcolor{blue}{\textbf{Output Termination:} Let $G$ be the underlying graph of $\mathcal{A}$. If $\delta(q, \texttt{true} = \tau) = (q', \sigma, b)$ is defined, let $C$ and $C'$ be the weakly\footnote{check} connected components $G$ that contain $q$ and $q'$, respectively. Then $C$ and $C'$ must be different components of $G$ and $C'$ must be acyclic. Further, for all states $q'\in C'$, $q'\in Q_{non}$ (i.e. no further states take in any input).} 
\end{itemize}

\subsubsection{Path probabilities}

\begin{defn}
	Two input sequences $\rho, \sigma\in \RR^*\times \{\tau\}$ are \textbf{adjacent} if $|\rho| = |\sigma|, \rho[|\rho|] = \sigma[|\rho|] = \tau$ and $\forall i < |\rho|, |\rho[i]-\sigma[i]| \leq 1$. 
\end{defn}

Note that path probabilities are exactly the same as in DiPAs, with the exception that a ITDiPA takes a $\texttt{input} = \tau$ transition with probability 1 if the input is $\tau$ and 0 otherwise.

\begin{defn}
	A ITDiPA $\mathcal{A}$ is \textbf{$d\epsilon$-differentially private} if for all adjacent $\rho, \sigma \in \RR^*\times \{\tau\}$, $\PP[d\epsilon, \rho] \leq e^{d\epsilon} \PP[\epsilon, \sigma]$.
\end{defn}

\subsubsection{Wellformedness}

For a ITDiPA $\mathcal{A}$, let $DiPA(\mathcal{A})$ be the DiPA created from $\mathcal{A}$ by removing all transitions with guard $\texttt{input} = \tau$ from the automaton and then removing all newly unreachable components. 

Then $\mathcal{A}$ is well-formed if and only if $DiPA(\mathcal{A})$.

\subsection{Differential Privacy}

\begin{thm}
	A ITDiPA $\mathcal{A}$ is well-formed if and only if it is differentially private.
\end{thm}

\begin{proof}
	Consider two paths 
\end{proof}

\bibliography{./dipalibrary}

\end{document} 